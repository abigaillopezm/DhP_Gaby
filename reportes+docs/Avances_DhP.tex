% Options for packages loaded elsewhere
% Options for packages loaded elsewhere
\PassOptionsToPackage{unicode}{hyperref}
\PassOptionsToPackage{hyphens}{url}
\PassOptionsToPackage{dvipsnames,svgnames,x11names}{xcolor}
%
\documentclass[
  11pt,
  letterpaper,
  DIV=11,
  numbers=noendperiod]{scrartcl}
\usepackage{xcolor}
\usepackage[top=2cm,left=2.5cm,right=2.5cm,bottom=2cm]{geometry}
\usepackage{amsmath,amssymb}
\setcounter{secnumdepth}{-\maxdimen} % remove section numbering
\usepackage{iftex}
\ifPDFTeX
  \usepackage[T1]{fontenc}
  \usepackage[utf8]{inputenc}
  \usepackage{textcomp} % provide euro and other symbols
\else % if luatex or xetex
  \usepackage{unicode-math} % this also loads fontspec
  \defaultfontfeatures{Scale=MatchLowercase}
  \defaultfontfeatures[\rmfamily]{Ligatures=TeX,Scale=1}
\fi
\usepackage{lmodern}
\ifPDFTeX\else
  % xetex/luatex font selection
  \setmainfont[]{Poppins}
  \setmonofont[Scale=0.85]{Courier New}
\fi
% Use upquote if available, for straight quotes in verbatim environments
\IfFileExists{upquote.sty}{\usepackage{upquote}}{}
\IfFileExists{microtype.sty}{% use microtype if available
  \usepackage[]{microtype}
  \UseMicrotypeSet[protrusion]{basicmath} % disable protrusion for tt fonts
}{}
\makeatletter
\@ifundefined{KOMAClassName}{% if non-KOMA class
  \IfFileExists{parskip.sty}{%
    \usepackage{parskip}
  }{% else
    \setlength{\parindent}{0pt}
    \setlength{\parskip}{6pt plus 2pt minus 1pt}}
}{% if KOMA class
  \KOMAoptions{parskip=half}}
\makeatother
% Make \paragraph and \subparagraph free-standing
\makeatletter
\ifx\paragraph\undefined\else
  \let\oldparagraph\paragraph
  \renewcommand{\paragraph}{
    \@ifstar
      \xxxParagraphStar
      \xxxParagraphNoStar
  }
  \newcommand{\xxxParagraphStar}[1]{\oldparagraph*{#1}\mbox{}}
  \newcommand{\xxxParagraphNoStar}[1]{\oldparagraph{#1}\mbox{}}
\fi
\ifx\subparagraph\undefined\else
  \let\oldsubparagraph\subparagraph
  \renewcommand{\subparagraph}{
    \@ifstar
      \xxxSubParagraphStar
      \xxxSubParagraphNoStar
  }
  \newcommand{\xxxSubParagraphStar}[1]{\oldsubparagraph*{#1}\mbox{}}
  \newcommand{\xxxSubParagraphNoStar}[1]{\oldsubparagraph{#1}\mbox{}}
\fi
\makeatother

\usepackage{color}
\usepackage{fancyvrb}
\newcommand{\VerbBar}{|}
\newcommand{\VERB}{\Verb[commandchars=\\\{\}]}
\DefineVerbatimEnvironment{Highlighting}{Verbatim}{commandchars=\\\{\}}
% Add ',fontsize=\small' for more characters per line
\usepackage{framed}
\definecolor{shadecolor}{RGB}{248,248,248}
\newenvironment{Shaded}{\begin{snugshade}}{\end{snugshade}}
\newcommand{\AlertTok}[1]{\textcolor[rgb]{0.94,0.16,0.16}{#1}}
\newcommand{\AnnotationTok}[1]{\textcolor[rgb]{0.56,0.35,0.01}{\textbf{\textit{#1}}}}
\newcommand{\AttributeTok}[1]{\textcolor[rgb]{0.13,0.29,0.53}{#1}}
\newcommand{\BaseNTok}[1]{\textcolor[rgb]{0.00,0.00,0.81}{#1}}
\newcommand{\BuiltInTok}[1]{#1}
\newcommand{\CharTok}[1]{\textcolor[rgb]{0.31,0.60,0.02}{#1}}
\newcommand{\CommentTok}[1]{\textcolor[rgb]{0.56,0.35,0.01}{\textit{#1}}}
\newcommand{\CommentVarTok}[1]{\textcolor[rgb]{0.56,0.35,0.01}{\textbf{\textit{#1}}}}
\newcommand{\ConstantTok}[1]{\textcolor[rgb]{0.56,0.35,0.01}{#1}}
\newcommand{\ControlFlowTok}[1]{\textcolor[rgb]{0.13,0.29,0.53}{\textbf{#1}}}
\newcommand{\DataTypeTok}[1]{\textcolor[rgb]{0.13,0.29,0.53}{#1}}
\newcommand{\DecValTok}[1]{\textcolor[rgb]{0.00,0.00,0.81}{#1}}
\newcommand{\DocumentationTok}[1]{\textcolor[rgb]{0.56,0.35,0.01}{\textbf{\textit{#1}}}}
\newcommand{\ErrorTok}[1]{\textcolor[rgb]{0.64,0.00,0.00}{\textbf{#1}}}
\newcommand{\ExtensionTok}[1]{#1}
\newcommand{\FloatTok}[1]{\textcolor[rgb]{0.00,0.00,0.81}{#1}}
\newcommand{\FunctionTok}[1]{\textcolor[rgb]{0.13,0.29,0.53}{\textbf{#1}}}
\newcommand{\ImportTok}[1]{#1}
\newcommand{\InformationTok}[1]{\textcolor[rgb]{0.56,0.35,0.01}{\textbf{\textit{#1}}}}
\newcommand{\KeywordTok}[1]{\textcolor[rgb]{0.13,0.29,0.53}{\textbf{#1}}}
\newcommand{\NormalTok}[1]{#1}
\newcommand{\OperatorTok}[1]{\textcolor[rgb]{0.81,0.36,0.00}{\textbf{#1}}}
\newcommand{\OtherTok}[1]{\textcolor[rgb]{0.56,0.35,0.01}{#1}}
\newcommand{\PreprocessorTok}[1]{\textcolor[rgb]{0.56,0.35,0.01}{\textit{#1}}}
\newcommand{\RegionMarkerTok}[1]{#1}
\newcommand{\SpecialCharTok}[1]{\textcolor[rgb]{0.81,0.36,0.00}{\textbf{#1}}}
\newcommand{\SpecialStringTok}[1]{\textcolor[rgb]{0.31,0.60,0.02}{#1}}
\newcommand{\StringTok}[1]{\textcolor[rgb]{0.31,0.60,0.02}{#1}}
\newcommand{\VariableTok}[1]{\textcolor[rgb]{0.00,0.00,0.00}{#1}}
\newcommand{\VerbatimStringTok}[1]{\textcolor[rgb]{0.31,0.60,0.02}{#1}}
\newcommand{\WarningTok}[1]{\textcolor[rgb]{0.56,0.35,0.01}{\textbf{\textit{#1}}}}

\usepackage{longtable,booktabs,array}
\usepackage{calc} % for calculating minipage widths
% Correct order of tables after \paragraph or \subparagraph
\usepackage{etoolbox}
\makeatletter
\patchcmd\longtable{\par}{\if@noskipsec\mbox{}\fi\par}{}{}
\makeatother
% Allow footnotes in longtable head/foot
\IfFileExists{footnotehyper.sty}{\usepackage{footnotehyper}}{\usepackage{footnote}}
\makesavenoteenv{longtable}
\usepackage{graphicx}
\makeatletter
\newsavebox\pandoc@box
\newcommand*\pandocbounded[1]{% scales image to fit in text height/width
  \sbox\pandoc@box{#1}%
  \Gscale@div\@tempa{\textheight}{\dimexpr\ht\pandoc@box+\dp\pandoc@box\relax}%
  \Gscale@div\@tempb{\linewidth}{\wd\pandoc@box}%
  \ifdim\@tempb\p@<\@tempa\p@\let\@tempa\@tempb\fi% select the smaller of both
  \ifdim\@tempa\p@<\p@\scalebox{\@tempa}{\usebox\pandoc@box}%
  \else\usebox{\pandoc@box}%
  \fi%
}
% Set default figure placement to htbp
\def\fps@figure{htbp}
\makeatother





\setlength{\emergencystretch}{3em} % prevent overfull lines

\providecommand{\tightlist}{%
  \setlength{\itemsep}{0pt}\setlength{\parskip}{0pt}}



 


\KOMAoption{captions}{tableheading}
\makeatletter
\@ifpackageloaded{tcolorbox}{}{\usepackage[skins,breakable]{tcolorbox}}
\@ifpackageloaded{fontawesome5}{}{\usepackage{fontawesome5}}
\definecolor{quarto-callout-color}{HTML}{909090}
\definecolor{quarto-callout-note-color}{HTML}{0758E5}
\definecolor{quarto-callout-important-color}{HTML}{CC1914}
\definecolor{quarto-callout-warning-color}{HTML}{EB9113}
\definecolor{quarto-callout-tip-color}{HTML}{00A047}
\definecolor{quarto-callout-caution-color}{HTML}{FC5300}
\definecolor{quarto-callout-color-frame}{HTML}{acacac}
\definecolor{quarto-callout-note-color-frame}{HTML}{4582ec}
\definecolor{quarto-callout-important-color-frame}{HTML}{d9534f}
\definecolor{quarto-callout-warning-color-frame}{HTML}{f0ad4e}
\definecolor{quarto-callout-tip-color-frame}{HTML}{02b875}
\definecolor{quarto-callout-caution-color-frame}{HTML}{fd7e14}
\makeatother
\makeatletter
\@ifpackageloaded{caption}{}{\usepackage{caption}}
\AtBeginDocument{%
\ifdefined\contentsname
  \renewcommand*\contentsname{Table of contents}
\else
  \newcommand\contentsname{Table of contents}
\fi
\ifdefined\listfigurename
  \renewcommand*\listfigurename{List of Figures}
\else
  \newcommand\listfigurename{List of Figures}
\fi
\ifdefined\listtablename
  \renewcommand*\listtablename{List of Tables}
\else
  \newcommand\listtablename{List of Tables}
\fi
\ifdefined\figurename
  \renewcommand*\figurename{Figure}
\else
  \newcommand\figurename{Figure}
\fi
\ifdefined\tablename
  \renewcommand*\tablename{Table}
\else
  \newcommand\tablename{Table}
\fi
}
\@ifpackageloaded{float}{}{\usepackage{float}}
\floatstyle{ruled}
\@ifundefined{c@chapter}{\newfloat{codelisting}{h}{lop}}{\newfloat{codelisting}{h}{lop}[chapter]}
\floatname{codelisting}{Listing}
\newcommand*\listoflistings{\listof{codelisting}{List of Listings}}
\makeatother
\makeatletter
\makeatother
\makeatletter
\@ifpackageloaded{caption}{}{\usepackage{caption}}
\@ifpackageloaded{subcaption}{}{\usepackage{subcaption}}
\makeatother
\makeatletter
\@ifpackageloaded{tcolorbox}{}{\usepackage[skins,breakable]{tcolorbox}}
\makeatother
\makeatletter
\@ifundefined{shadecolor}{\definecolor{shadecolor}{HTML}{cccccc}}{}
\makeatother
\makeatletter
\makeatother
\makeatletter
\ifdefined\Shaded\renewenvironment{Shaded}{\begin{tcolorbox}[breakable, frame hidden, borderline west={3pt}{0pt}{shadecolor}, interior hidden, enhanced, boxrule=0pt, sharp corners]}{\end{tcolorbox}}\fi
\makeatother
\usepackage{bookmark}
\IfFileExists{xurl.sty}{\usepackage{xurl}}{} % add URL line breaks if available
\urlstyle{same}
\hypersetup{
  pdftitle={Avance en Trabajo Doctoral: Caracterización climática},
  pdfauthor={AL},
  colorlinks=true,
  linkcolor={blue},
  filecolor={Maroon},
  citecolor={Blue},
  urlcolor={Blue},
  pdfcreator={LaTeX via pandoc}}


\title{Avance en Trabajo Doctoral: Caracterización climática}
\author{AL}
\date{2026-08-06}
\begin{document}
\maketitle


\begin{tcolorbox}[enhanced jigsaw, breakable, left=2mm, arc=.35mm, colbacktitle=quarto-callout-tip-color!10!white, colframe=quarto-callout-tip-color-frame, leftrule=.75mm, opacityback=0, toptitle=1mm, opacitybacktitle=0.6, bottomtitle=1mm, toprule=.15mm, titlerule=0mm, rightrule=.15mm, title=\textcolor{quarto-callout-tip-color}{\faLightbulb}\hspace{0.5em}{repositorio del trabajo}, colback=white, bottomrule=.15mm, coltitle=black]

\begin{figure}[H]

{\centering \pandocbounded{\includegraphics[keepaspectratio]{Avances_DhP_files/mediabag/DhP_Gaby.git}}

}

\caption{Puede visitarse el repositorio del trabajo en este link}

\end{figure}%

\end{tcolorbox}

\section{Sobre el área de estudio}\label{sobre-el-uxe1rea-de-estudio}

\begin{tcolorbox}[enhanced jigsaw, breakable, left=2mm, arc=.35mm, colbacktitle=quarto-callout-important-color!10!white, colframe=quarto-callout-important-color-frame, leftrule=.75mm, opacityback=0, toptitle=1mm, opacitybacktitle=0.6, bottomtitle=1mm, toprule=.15mm, titlerule=0mm, rightrule=.15mm, title=\textcolor{quarto-callout-important-color}{\faExclamation}\hspace{0.5em}{Descripción de la sección}, colback=white, bottomrule=.15mm, coltitle=black]

En esta sección se encuentra la información obtenida del análisis
climático de la zona de estudio

\end{tcolorbox}

El área de estudio se ubica entre las coordenadas {[}insertar
coordenadas del área de estudio{]}, considerando como punto medio del
dominio el Pico de Orizaba. Al hacer un conteo de las estaciones
meteorológicas que se encuentran en la zona, se enoncontró la existencia
de 106 de estas; de las cuales X se encuentran en el estado de Veracruz,
X en Puebla y X en Tlaxcala. Su posición se muestra en el mapa 1.
\pandocbounded{\includegraphics[keepaspectratio]{../data/AnalisisGeografico/Analisis_Geografico_estaciones+ubi.png}}

\section{Caracterización
climática}\label{caracterizaciuxf3n-climuxe1tica}

Usando datos del INEGI, se hizo la caracterización climática obteniendo
los resultados que se se muestran a continuación\footnote{El script con
  el que hicieron los mapas se encuentran en la sección de los
  \hyperref[script-mapas-de-caracterizaciuxf3n-climuxe1tica]{Anexos}}.

\subsection{Tipos de clima}\label{tipos-de-clima}

\begin{center}
\includegraphics[width=0.85\linewidth,height=\textheight,keepaspectratio]{../outputs/00.CLIMAS.png}
\end{center}
\begin{center}
\includegraphics[width=0.7\linewidth,height=\textheight,keepaspectratio]{../outputs/00.CLIM_HUM.png}
\end{center}

\subsection{Edafología}\label{edafologuxeda}

\begin{figure}[H]

{\centering \includegraphics[width=0.85\linewidth,height=\textheight,keepaspectratio]{../outputs/00.EDAFO.png}

}

\caption{Edafología de la zona de estudio}

\end{figure}%

\subsection{Hidrografía e
hidrología}\label{hidrografuxeda-e-hidrologuxeda}

\begin{figure}[H]

{\centering \pandocbounded{\includegraphics[keepaspectratio]{../outputs/00.HIDRO.png}}

}

\caption{Hidrografía y regiones hidrológicas de la zona de estudio}

\end{figure}%

\section{Sobre los datos utilizados}\label{sobre-los-datos-utilizados}

\begin{tcolorbox}[enhanced jigsaw, breakable, left=2mm, arc=.35mm, colbacktitle=quarto-callout-important-color!10!white, colframe=quarto-callout-important-color-frame, leftrule=.75mm, opacityback=0, toptitle=1mm, opacitybacktitle=0.6, bottomtitle=1mm, toprule=.15mm, titlerule=0mm, rightrule=.15mm, title=\textcolor{quarto-callout-important-color}{\faExclamation}\hspace{0.5em}{Descripción de la sección}, colback=white, bottomrule=.15mm, coltitle=black]

En esta sección se encuentran los métodos y ajustes hehcos en los datos
por analizar

\end{tcolorbox}

\subsection{Sobre los datos en malla (artículo
Jaime)}\label{sobre-los-datos-en-malla-artuxedculo-jaime}

\begin{enumerate}
\def\labelenumi{\arabic{enumi}.}
\tightlist
\item
  Se descargaron los datos de un año para hacer el análisis exploratorio
  y planificar el flujo de trabajo
\item
  Se decidió unir los archivos tif a un solo archivo NetCDF con el fin
  de poder optimizar el trabajo (Revisar los
  \hyperref[scripts-de-trabajo-con-mallas-nc-y-tif]{anexos} para ver los
  scripts con los que se hizo este trabajo).
\item
  Se hizo el ploteo exploratorio de la malla.
\end{enumerate}

\subsection{Sobre los datos del SMN}\label{sobre-los-datos-del-smn}

Para poder tener la certeza de la validez de los datos antes trabajados,
se hizo la comparación de los datos de precipitación y temperatura para
10 puntos (estaciones) de la malla. Estas estaciones se describen en la
Tabla 1

\%\% insertar la tabla \%\%

\section{Anexos}\label{anexos}

\subsection{Script: mapas de caracterización
climática}\label{script-mapas-de-caracterizaciuxf3n-climuxe1tica}

\begin{Shaded}
\begin{Highlighting}[numbers=left,,]
\CommentTok{\#Primero, fijar el directorio de trabajo}
\FunctionTok{setwd}\NormalTok{(}\StringTok{"\textasciitilde{}/001\_DhPGaby\_v2/dhp\_gaby{-}1"}\NormalTok{)}

\CommentTok{\#cargar las librerias}
\FunctionTok{library}\NormalTok{(sf)}
\FunctionTok{library}\NormalTok{(ggplot2)}
\FunctionTok{library}\NormalTok{(RColorBrewer)}
\FunctionTok{library}\NormalTok{(colorspace)}

\CommentTok{\#Ruta de los datos y definición de variables}
\NormalTok{path\_climas }\OtherTok{\textless{}{-}} \StringTok{"data/AnalisisGeografico/Climas\_1mgw/clima1mgw.shp"}
\NormalTok{path\_climhum }\OtherTok{\textless{}{-}} \StringTok{"data/AnalisisGeografico/humed4mgw/humed4mgw.shp"}
\NormalTok{path\_edafo }\OtherTok{\textless{}{-}} \StringTok{"data/AnalisisGeografico/edafologoia\_4mgw/edafo4mgw.shp"}
\NormalTok{path\_rios }\OtherTok{\textless{}{-}} \StringTok{"data/AnalisisGeografico/Regiones{-}hidrologicas\_4mgw/hidro4mgw.shp"}
\NormalTok{path\_hidro }\OtherTok{\textless{}{-}} \StringTok{"data/AnalisisGeografico/Regiones{-}hidrologicas\_250kgw/rh250kgw.shp"}
\NormalTok{path\_suelo7 }\OtherTok{\textless{}{-}} \StringTok{"data/AnalisisGeografico/usv250s7gw\_INEGI\_SERIE7/usv250s7gw.shp"}
\NormalTok{path\_pico }\OtherTok{\textless{}{-}} \StringTok{"data/z.AREA/Area\_Metodo/Poligono\_ANP\_Pico.shp"}
\NormalTok{path\_estat }\OtherTok{\textless{}{-}} \StringTok{"data/AnalisisGeografico/DivisonEstatal\_dest23gw/dest23gw.shp"}
\NormalTok{path\_domKML }\OtherTok{\textless{}{-}} \StringTok{"data/Dom\_50km.kml"}

\NormalTok{climas }\OtherTok{\textless{}{-}} \FunctionTok{st\_read}\NormalTok{(path\_climas)}
\NormalTok{climhum }\OtherTok{\textless{}{-}} \FunctionTok{st\_read}\NormalTok{(path\_climhum)}
\NormalTok{edafo }\OtherTok{\textless{}{-}} \FunctionTok{st\_read}\NormalTok{(path\_edafo)}
\NormalTok{rios }\OtherTok{\textless{}{-}} \FunctionTok{st\_read}\NormalTok{(path\_rios)}
\NormalTok{hidro }\OtherTok{\textless{}{-}} \FunctionTok{st\_read}\NormalTok{(path\_hidro)}
\NormalTok{suelo7 }\OtherTok{\textless{}{-}} \FunctionTok{st\_read}\NormalTok{(path\_suelo7)}
\NormalTok{ANPpico }\OtherTok{\textless{}{-}} \FunctionTok{st\_read}\NormalTok{(path\_pico)}
\NormalTok{estat }\OtherTok{\textless{}{-}} \FunctionTok{st\_read}\NormalTok{(path\_estat)}
\CommentTok{\#suelo7v2 \textless{}{-} st\_make\_valid(suelo7) \#corrige error del shape de uso de suelo}
\CommentTok{\#Errores por invalidez de datos}
\NormalTok{dominio}\OtherTok{\textless{}{-}} \FunctionTok{st\_read}\NormalTok{(path\_domKML)}

\CommentTok{\#ploteo de prueba}
\FunctionTok{plot}\NormalTok{(edafo) }\CommentTok{\#plotea todas las cosas}

\DocumentationTok{\#\#\#\#\#\#\#}
\CommentTok{\# Recortar utilizando un kml (interseccionar)}
\NormalTok{climas\_dom }\OtherTok{\textless{}{-}} \FunctionTok{st\_intersection}\NormalTok{(climas, dominio) }\CommentTok{\#CLIMA\_TIPO}
\NormalTok{climhum\_dom }\OtherTok{\textless{}{-}} \FunctionTok{st\_intersection}\NormalTok{(climhum, dominio) }\CommentTok{\#TIPO}
\NormalTok{edafo\_dom }\OtherTok{\textless{}{-}} \FunctionTok{st\_intersection}\NormalTok{(edafo, dominio) }\CommentTok{\#UNIDAD SUE}
\NormalTok{rios\_dom }\OtherTok{\textless{}{-}} \FunctionTok{st\_intersection}\NormalTok{(rios, dominio) }\CommentTok{\#NOMRES}
\NormalTok{hidro\_dom }\OtherTok{\textless{}{-}} \FunctionTok{st\_intersection}\NormalTok{(hidro, dominio) }\CommentTok{\#NOMBRE}
\NormalTok{estat\_dom }\OtherTok{\textless{}{-}} \FunctionTok{st\_intersection}\NormalTok{(estat, dominio) }\CommentTok{\#}
\CommentTok{\#suelo7\_dom \textless{}{-} st\_intersection(suelo7, dominio) \#corregir error de validez}

\CommentTok{\#ploteo avanzado}
\DocumentationTok{\#\#\#\#\#}
\CommentTok{\#CLIMAS}
\NormalTok{n\_clases }\OtherTok{\textless{}{-}} \FunctionTok{length}\NormalTok{(}\FunctionTok{unique}\NormalTok{(climas\_dom}\SpecialCharTok{$}\NormalTok{CLIMA\_TIPO))}
\NormalTok{colores }\OtherTok{\textless{}{-}} \FunctionTok{qualitative\_hcl}\NormalTok{(n\_clases, }\AttributeTok{palette =} \StringTok{"Dark 3"}\NormalTok{)}
\FunctionTok{ggplot}\NormalTok{() }\SpecialCharTok{+}
  \FunctionTok{geom\_sf}\NormalTok{(}\AttributeTok{data =}\NormalTok{ climas\_dom, }\FunctionTok{aes}\NormalTok{(}\AttributeTok{fill =}\NormalTok{ CLIMA\_TIPO)) }\SpecialCharTok{+}
  \FunctionTok{geom\_sf}\NormalTok{(}\AttributeTok{data =}\NormalTok{ ANPpico) }\SpecialCharTok{+}
  \FunctionTok{annotate}\NormalTok{(}\StringTok{"text"}\NormalTok{, }\AttributeTok{x =} \SpecialCharTok{{-}}\FloatTok{97.28}\NormalTok{, }\AttributeTok{y =} \FloatTok{19.05}\NormalTok{, }\AttributeTok{label =} \StringTok{"ANP"}\NormalTok{, }\AttributeTok{size =} \DecValTok{5}\NormalTok{) }\SpecialCharTok{+}
  \FunctionTok{scale\_fill\_manual}\NormalTok{(}\AttributeTok{values =}\NormalTok{ colores) }\SpecialCharTok{+}
  \FunctionTok{theme\_light}\NormalTok{(}\AttributeTok{base\_size =} \DecValTok{13}\NormalTok{) }\SpecialCharTok{+}
  \FunctionTok{ggtitle}\NormalTok{(}\StringTok{"Tipos de clima"}\NormalTok{) }\SpecialCharTok{+} \FunctionTok{xlab}\NormalTok{(}\StringTok{"Longitud"}\NormalTok{) }\SpecialCharTok{+} \FunctionTok{ylab}\NormalTok{(}\StringTok{"Latitud"}\NormalTok{) }\SpecialCharTok{+}
  \FunctionTok{labs}\NormalTok{(}\AttributeTok{fill =} \StringTok{"Leyenda"}\NormalTok{,}
       \AttributeTok{caption =} \StringTok{"Fuente: García (1998), CONABIO (2001).}
\StringTok{       Climas de México, escala 1:1,000,000."}\NormalTok{) }\SpecialCharTok{+}
  \FunctionTok{theme}\NormalTok{(}\AttributeTok{plot.caption =} \FunctionTok{element\_text}\NormalTok{(}\AttributeTok{size =} \DecValTok{7}\NormalTok{, }\AttributeTok{color =} \StringTok{"gray40"}\NormalTok{, }\AttributeTok{face =} \StringTok{"italic"}\NormalTok{, }\AttributeTok{hjust =} \DecValTok{1}\NormalTok{))}

\FunctionTok{ggsave}\NormalTok{(}\StringTok{"outputs/00.CLIMAS.png"}\NormalTok{, }\AttributeTok{width =} \DecValTok{11}\NormalTok{, }\AttributeTok{height =} \DecValTok{8}\NormalTok{, }\AttributeTok{dpi =} \DecValTok{300}\NormalTok{)}

\DocumentationTok{\#\#\#\#\#}
\CommentTok{\# CLIMAS HUMEDAD}
\FunctionTok{ggplot}\NormalTok{() }\SpecialCharTok{+}
  \FunctionTok{geom\_sf}\NormalTok{(}\AttributeTok{data =}\NormalTok{ climhum\_dom, }\FunctionTok{aes}\NormalTok{(}\AttributeTok{fill =}\NormalTok{ TIPO))}\SpecialCharTok{+} \CommentTok{\#nombre de la variable, se ve con head()}
  \FunctionTok{geom\_sf}\NormalTok{(}\AttributeTok{data =}\NormalTok{ ANPpico) }\SpecialCharTok{+}
  \FunctionTok{annotate}\NormalTok{(}\StringTok{"text"}\NormalTok{, }\AttributeTok{x =} \SpecialCharTok{{-}}\FloatTok{97.28}\NormalTok{, }\AttributeTok{y =} \FloatTok{19.05}\NormalTok{, }\AttributeTok{label =} \StringTok{"ANP"}\NormalTok{, }\AttributeTok{size =} \DecValTok{5}\NormalTok{) }\SpecialCharTok{+}
  \FunctionTok{scale\_fill\_brewer}\NormalTok{(}\AttributeTok{palette =} \StringTok{"BrBG"}\NormalTok{) }\SpecialCharTok{+}
  \CommentTok{\#geom\_contour(data = estat\_dom, color = "white")}
  \FunctionTok{theme\_light}\NormalTok{(}\AttributeTok{base\_size =} \DecValTok{13}\NormalTok{) }\SpecialCharTok{+}
  \FunctionTok{ggtitle}\NormalTok{(}\StringTok{"Tipos de clima por rango de humedad"}\NormalTok{) }\SpecialCharTok{+} \FunctionTok{xlab}\NormalTok{(}\StringTok{"Longitud"}\NormalTok{) }\SpecialCharTok{+} \FunctionTok{ylab}\NormalTok{(}\StringTok{"Latitud"}\NormalTok{) }\SpecialCharTok{+}
  \FunctionTok{labs}\NormalTok{(}\AttributeTok{fill =} \StringTok{"Leyenda"}\NormalTok{,}
       \AttributeTok{caption =} \StringTok{"Fuente: García (1990), Atlas Nacional de México.}
\StringTok{       Rangos de humedad, escala 1:4,000,000 (UNAM, IGg)."}\NormalTok{) }\SpecialCharTok{+}
  \FunctionTok{theme}\NormalTok{(}\AttributeTok{plot.caption =} \FunctionTok{element\_text}\NormalTok{(}\AttributeTok{size =} \DecValTok{7}\NormalTok{, }\AttributeTok{color =} \StringTok{"gray40"}\NormalTok{, }\AttributeTok{face =} \StringTok{"italic"}\NormalTok{, }\AttributeTok{hjust =} \DecValTok{1}\NormalTok{))}
\FunctionTok{ggsave}\NormalTok{(}\StringTok{"outputs/00.CLIM\_HUM.png"}\NormalTok{, }\AttributeTok{width =} \DecValTok{11}\NormalTok{, }\AttributeTok{height =} \DecValTok{8}\NormalTok{, }\AttributeTok{dpi =} \DecValTok{300}\NormalTok{)}

\DocumentationTok{\#\#\#\#\#}
\CommentTok{\# EDAFOLOGIA}
\FunctionTok{ggplot}\NormalTok{() }\SpecialCharTok{+}
  \FunctionTok{geom\_sf}\NormalTok{(}\AttributeTok{data =}\NormalTok{ edafo\_dom, }\FunctionTok{aes}\NormalTok{(}\AttributeTok{fill =}\NormalTok{ UNIDAD\_SUE))}\SpecialCharTok{+} \CommentTok{\#nombre de la variable, se ve con head()}
  \FunctionTok{geom\_sf}\NormalTok{(}\AttributeTok{data =}\NormalTok{ ANPpico) }\SpecialCharTok{+}
  \FunctionTok{annotate}\NormalTok{(}\StringTok{"text"}\NormalTok{, }\AttributeTok{x =} \SpecialCharTok{{-}}\FloatTok{97.28}\NormalTok{, }\AttributeTok{y =} \FloatTok{19.05}\NormalTok{, }\AttributeTok{label =} \StringTok{"ANP"}\NormalTok{, }\AttributeTok{size =} \DecValTok{5}\NormalTok{) }\SpecialCharTok{+}
  \FunctionTok{scale\_fill\_brewer}\NormalTok{(}\AttributeTok{palette =} \StringTok{"Set3"}\NormalTok{) }\SpecialCharTok{+}
  \CommentTok{\#geom\_contour(data = estat\_dom, color = "white")}
  \FunctionTok{theme\_light}\NormalTok{(}\AttributeTok{base\_size =} \DecValTok{13}\NormalTok{) }\SpecialCharTok{+}
  \FunctionTok{ggtitle}\NormalTok{(}\StringTok{"Edafología"}\NormalTok{) }\SpecialCharTok{+} \FunctionTok{xlab}\NormalTok{(}\StringTok{"Longitud"}\NormalTok{) }\SpecialCharTok{+} \FunctionTok{ylab}\NormalTok{(}\StringTok{"Latitud"}\NormalTok{) }\SpecialCharTok{+}
  \FunctionTok{labs}\NormalTok{(}\AttributeTok{fill =} \StringTok{"Leyenda"}\NormalTok{,}
       \AttributeTok{caption =} \StringTok{"Fuente: SEMARNAP (1998). Mapa de suelos dominantes}
\StringTok{       de la República Mexicana. Primera aproximación 1996,}
\StringTok{       escala 1:4,000,000."}\NormalTok{) }\SpecialCharTok{+}
  \FunctionTok{theme}\NormalTok{(}\AttributeTok{plot.caption =} \FunctionTok{element\_text}\NormalTok{(}\AttributeTok{size =} \DecValTok{7}\NormalTok{, }\AttributeTok{color =} \StringTok{"gray40"}\NormalTok{, }\AttributeTok{face =} \StringTok{"italic"}\NormalTok{, }\AttributeTok{hjust =} \DecValTok{1}\NormalTok{))}
\FunctionTok{ggsave}\NormalTok{(}\StringTok{"outputs/00.EDAFO.png"}\NormalTok{, }\AttributeTok{width =} \DecValTok{11}\NormalTok{, }\AttributeTok{height =} \DecValTok{8}\NormalTok{, }\AttributeTok{dpi =} \DecValTok{300}\NormalTok{)}


\DocumentationTok{\#\#\#\#\#}
\CommentTok{\#CUENCAS y RIOS}
\FunctionTok{ggplot}\NormalTok{() }\SpecialCharTok{+}
  \FunctionTok{geom\_sf}\NormalTok{(}\AttributeTok{data =}\NormalTok{ hidro\_dom, }\FunctionTok{aes}\NormalTok{(}\AttributeTok{fill =}\NormalTok{ NOMBRE))}\SpecialCharTok{+} \CommentTok{\#nombre de la variable, se ve con head()}
  \FunctionTok{geom\_sf}\NormalTok{(}\AttributeTok{data =}\NormalTok{ ANPpico) }\SpecialCharTok{+}
  \FunctionTok{annotate}\NormalTok{(}\StringTok{"text"}\NormalTok{, }\AttributeTok{x =} \SpecialCharTok{{-}}\FloatTok{97.28}\NormalTok{, }\AttributeTok{y =} \FloatTok{19.05}\NormalTok{, }\AttributeTok{label =} \StringTok{"ANP"}\NormalTok{, }\AttributeTok{size =} \DecValTok{5}\NormalTok{) }\SpecialCharTok{+}
  \FunctionTok{scale\_fill\_brewer}\NormalTok{(}\AttributeTok{name =} \StringTok{"Dark2"}\NormalTok{)}\SpecialCharTok{+}
  \FunctionTok{geom\_sf}\NormalTok{(}\AttributeTok{data=}\NormalTok{rios\_dom, }\AttributeTok{color =} \StringTok{"blue"}\NormalTok{)}\SpecialCharTok{+}
  \FunctionTok{geom\_sf\_text}\NormalTok{(}\AttributeTok{data =}\NormalTok{ rios\_dom, }\FunctionTok{aes}\NormalTok{(}\AttributeTok{label =}\NormalTok{ NOMBRES), }\AttributeTok{size =} \DecValTok{3}\NormalTok{, }\AttributeTok{color =} \StringTok{"black"}\NormalTok{) }\SpecialCharTok{+}
  \FunctionTok{theme\_light}\NormalTok{(}\AttributeTok{base\_size =} \DecValTok{13}\NormalTok{) }\SpecialCharTok{+}
  \FunctionTok{ggtitle}\NormalTok{(}\StringTok{"Cuencas hidrológicas y ríos"}\NormalTok{) }\SpecialCharTok{+} \FunctionTok{xlab}\NormalTok{(}\StringTok{"Longitud"}\NormalTok{) }\SpecialCharTok{+} \FunctionTok{ylab}\NormalTok{(}\StringTok{"Latitud"}\NormalTok{) }\SpecialCharTok{+}
  \FunctionTok{labs}\NormalTok{(}\AttributeTok{fill =} \StringTok{"Leyenda"}\NormalTok{, }
      \AttributeTok{caption =} \StringTok{"Fuentes: CONAGUA (2009), Regiones Hidrológicas}
\StringTok{      (escala 1:250,000). Maderey{-}R \& Torres{-}Ruata (1998), Hidrografía, }
\StringTok{      Atlas Nacional de México (escala 1:4,000,000)."}\NormalTok{) }\SpecialCharTok{+}
  \FunctionTok{theme}\NormalTok{(}\AttributeTok{plot.caption =} \FunctionTok{element\_text}\NormalTok{(}\AttributeTok{size =} \DecValTok{7}\NormalTok{, }\AttributeTok{color =} \StringTok{"gray40"}\NormalTok{, }\AttributeTok{face =} \StringTok{"italic"}\NormalTok{, }\AttributeTok{hjust =} \DecValTok{1}\NormalTok{))}
\FunctionTok{ggsave}\NormalTok{(}\StringTok{"outputs/00.HIDRO.png"}\NormalTok{, }\AttributeTok{width =} \DecValTok{11}\NormalTok{, }\AttributeTok{height =} \DecValTok{8}\NormalTok{, }\AttributeTok{dpi =} \DecValTok{300}\NormalTok{)}
\end{Highlighting}
\end{Shaded}

\subsection{\texorpdfstring{Scripts de trabajo con mallas \emph{nc y
}tif}{Scripts de trabajo con mallas nc y tif}}\label{scripts-de-trabajo-con-mallas-nc-y-tif}

\subsubsection{Script: Conversión de archivos tif a
NetCDF}\label{script-conversiuxf3n-de-archivos-tif-a-netcdf}

\begin{Shaded}
\begin{Highlighting}[numbers=left,,]
\FunctionTok{mkdir}\NormalTok{ data/nc\_files51}
\ControlFlowTok{for}\NormalTok{ file }\KeywordTok{in}\NormalTok{ data/1951/}\PreprocessorTok{*}\NormalTok{.tif}\KeywordTok{;} \ControlFlowTok{do}
  \VariableTok{name}\OperatorTok{=}\VariableTok{$(}\FunctionTok{basename} \StringTok{"}\VariableTok{$file}\StringTok{"}\NormalTok{ .tif}\VariableTok{)}
  \ExtensionTok{gdal\_translate} \AttributeTok{{-}of}\NormalTok{ netCDF }\StringTok{"}\VariableTok{$file}\StringTok{"} \StringTok{"data/nc\_files51/}\VariableTok{$\{name\}}\StringTok{.nc"}
\ControlFlowTok{done}
\end{Highlighting}
\end{Shaded}

\subsubsection{Script: Corte de archivos tif al área de
estudio}\label{script-corte-de-archivos-tif-al-uxe1rea-de-estudio}

\begin{Shaded}
\begin{Highlighting}[numbers=left,,]
\FunctionTok{mkdir}\NormalTok{ data/clipped51}
\ControlFlowTok{for}\NormalTok{ file }\KeywordTok{in}\NormalTok{ data/nc\_files51/}\PreprocessorTok{*}\NormalTok{.nc}\KeywordTok{;} \ControlFlowTok{do}
  \ExtensionTok{cdo} \AttributeTok{{-}f}\NormalTok{ nc sellonlatbox,{-}98,95,17,20 }\StringTok{"}\VariableTok{$file}\StringTok{"} \StringTok{"data/clipped51/}\VariableTok{$(}\FunctionTok{basename} \VariableTok{$file)}\StringTok{"}
\ControlFlowTok{done}
\end{Highlighting}
\end{Shaded}

\subsubsection{\texorpdfstring{Script: unión de los archivos
\texttt{*.nc} a uno solo
(mergetime)}{Script: unión de los archivos *.nc a uno solo (mergetime)}}\label{script-uniuxf3n-de-los-archivos-.nc-a-uno-solo-mergetime}

\begin{Shaded}
\begin{Highlighting}[numbers=left,,]
\VariableTok{i}\OperatorTok{=}\NormalTok{0}
\ControlFlowTok{for}\NormalTok{ file }\KeywordTok{in}\NormalTok{ precip1951}\PreprocessorTok{*}\NormalTok{.nc}\KeywordTok{;} \ControlFlowTok{do}
  \VariableTok{date}\OperatorTok{=}\VariableTok{$(}\FunctionTok{date} \AttributeTok{{-}d} \StringTok{"1951{-}01{-}01 +}\VariableTok{$i}\StringTok{ days"}\NormalTok{ +\%Y{-}\%m{-}\%d}\VariableTok{)}
  \ExtensionTok{cdo}\NormalTok{ settaxis,}\VariableTok{$\{date\}}\NormalTok{,00:00:00,1day }\StringTok{"}\VariableTok{$file}\StringTok{"} \StringTok{"fixed\_}\VariableTok{$file}\StringTok{"}
  \KeywordTok{((}\VariableTok{i}\OperatorTok{++}\KeywordTok{))}
\ControlFlowTok{done}
\end{Highlighting}
\end{Shaded}





\end{document}
